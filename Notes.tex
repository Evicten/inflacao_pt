\documentclass{article}
\usepackage{graphicx} % Required for inserting images
\usepackage{url}
\usepackage{hyperref}
%\usepackage{natbib}
\usepackage{verbatim}
\usepackage{natbib}
\usepackage{amsmath} 

\usepackage[letterpaper,top=3cm,bottom=3cm,left=4cm,right=4cm,marginparwidth=1.75cm]{geometry}

\begin{document}

"In 2009/2010, 76 \% of the students in upper secondary schooling were enrolled in public institutions (72 \% in ‘regular’ public schools, 4 \% in TEIP schools) and 24 \% in private schools (19 \% in independent schools, 5 \% in government-dependent private schools; OECD 2012, p. 333; MEC 2012). The following year, 77 \% of the 377,389 higher education students were enrolled in public institutions, and the remaining 23 \% in private ones (Fonseca and Encarnac¸a˜o 2012)."



Jacob Østdal suggestion: See two individuals above the cutoff, see one public and private. If private grades are inflated, then the public school kid should have higher ability and do better at university.


Internal grade is 70\% and 30\% external grade.

CIF é classificao interna final. Antes dos 30\% da nota de exame. quando este é contabilizado, passa a ser CFD: classificacao final da disciplina.
Portanto, CIF é só nota interna.

Ajuste da nota:

Adjusted grade = CIF + diff\_escola



\end{document}