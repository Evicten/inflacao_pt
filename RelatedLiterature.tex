\documentclass{article}
\usepackage{graphicx} % Required for inserting images
\usepackage{natbib}
\usepackage{hyperref}


\title{Related Literature}
\author{Miguel Duarte}
\date{November 2024}

\begin{document}

\maketitle

Most relevant paper: \cite{silva2025public}

Teachers get an evaluation. If sufficient or negative, then they are not entitled to progression. "Good" allows for progression, "Muito Bom" allows for faster progression and "Excellent" for even faster progression. "Muito Bom" and "Excellent" grades have an upper bound into how many can be awarded in each HS.




\vspace{5pt}
\href{https://www.degruyter.com/document/doi/10.1515/9780791478004-007/html?srsltid=AfmBOoqNX5V2WbQb7DQAKwi0SG4xrgmtQYzNi9m22tpPYjjfakeslkoV}{Grade Inflation and Grade Variation: What’s All the Fuss About?}, by Harry Brighouse (2008)

cant access

\vspace{5pt} \href{https://www.iza.org/publications/dp/4051/individual-teacher-incentives-student-achievement-and-grade-inflation}{Individual Teacher Incentives, Student Achievement, and Grade Inflation}, by Pedro S. Martins (2009)

The authors investigate how tying teacher pay to student performance, as measured by internal school grades and external national exam results, impacted overall student achievement. The study finds that the \textbf{introduction of individual teacher incentives led to: A significant and sizable decline in student achievement}, particularly in terms of national exams; An increase in grade inflation, as evidenced by a \textbf{widening gap between internal school grades and external exam results}.
Key features of the reform include: 
Breaking up the single pay scale for teachers into two scales, making progression from the lower to the upper scale contingent on performance.

The authors suggest that the observed negative effects on student achievement may be due to:
Disruption of collaborative work among teachers due to the competitive nature of the new incentive system; 
Teachers focusing more on "teaching to the test" for internal assessments, which hold a higher weighting in the final student grade, rather than focusing on broader learning outcomes.
%This is summary from NotebookLM. How does the gap widen? Are the performances of teachers only based on internal grades? That makes no sense.


\vspace{5pt} \href{https://link.springer.com/article/10.1007/s10734-007-9097-x}{Evaluating Student Allocation in the Portuguese Public Higher Education System}, by Miguel Portela, Nelson Areal, Ana Carvalho, Artur Rodrigues, Carla Sá, Fernando Alexandre, and João Cerejeira (2008)

\vspace{5pt} \href{https://www.sciencedirect.com/science/article/pii/S0272775704000895}{Grade Inflation: An Issue for Higher Education?}, by David A. Sabotka (2004)

\vspace{5pt} \href{https://www.tandfonline.com/doi/abs/10.1080/15582159.2010.483918}{Does Competition Among Schools Encourage Grade Inflation?}, by Patrick Walsh (2010)

\vspace{5pt} \href{https://www.tandfonline.com/doi/full/10.1080/09695940500143811}{Grade Stability in a Criterion-Referenced Grading System: The Swedish Example}, by Christina Wikström (2005)

\vspace{5pt} \href{https://www.sciencedirect.com/science/article/pii/S0272775704000895}{Grade Inflation and School Competition: An Empirical Analysis Based on the Swedish Upper Secondary Schools}, by Christina Wikström and Magnus Wikström (2005)

\vspace{5pt} \href{https://onlinelibrary.wiley.com/doi/10.1111/j.1745-3984.2002.tb01133.x}{Grades and Test Scores: Accounting for Observed Differences}, by Warren W. Willingham, Judith M. Pollack, and Charles Lewis (2002)

Woodruff, D. J., \& Ziomek, R. L. (2004a). “Differential grading standards among high schools.” ACT Reseach Report Series, 2004–2.

Woodruff, D. J., \& Ziomek, R. L. (2004b). “High school grade inflation from 1991 to 2003.” ACT Reseach Report Series, 2004-4.



\newpage
%\bibliographystyle{plainnat}
\bibliographystyle{apalike}
\bibliography{references}


\end{document}