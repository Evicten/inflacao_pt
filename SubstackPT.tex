
\documentclass{article}
\usepackage{graphicx} % Required for inserting images
\usepackage{url}
\usepackage{hyperref}
%\usepackage{natbib}
\usepackage{verbatim}
\usepackage{natbib}
\usepackage{amsmath} 
\usepackage{xcolor}

\usepackage[letterpaper,top=3cm,bottom=3cm,left=4cm,right=4cm,marginparwidth=1.75cm]{geometry}

\begin{document}

\title{Disparidades no Acesso ao Ensino Superior em Portugal}
\author{Miguel Sousa Duarte\thanks{{\tiny{}Department of Economics, Copenhagen Business School. \textbf{E-mail}: \url{msd.eco@cbs.dk}.}},
Vicente Conde Mendes\thanks{{\tiny{}École Polytechnique Fédérale de Lausanne. \textbf{E-mail}: \url{vicente.c.mendes@gmail.com}.}}\\}

\date{\today}

\setlength{\parskip}{1em}
\maketitle

\section{Introdução}

%Embora os alunos das escolas privadas em Portugal tendam a superar os seus colegas das escolas públicas tanto nas classificações internas como nos exames nacionais, também apresentam uma maior discrepância entre as duas — ou seja, a diferença entre as notas atribuídas pelas escolas e os resultados nos exames é mais acentuada. Se o acesso ao ensino superior assentar em classificações internas inflacionadas, os alunos das escolas privadas poderão beneficiar de forma desproporcionada. Neste estudo, quantificamos a inflação de notas entre escolas e o seu impacto no acesso ao ensino superior.

A inflação de notas no ensino secundário em Portugal é um fenómeno muito discutido informalmente, sobre o qual todos têm uma opinião. É um fenómeno que gera emoções porque, acontecendo, é uma enorme injustiça, devido ao impacto das notas internas no acesso ao Ensino Superior no nosso país. Será que a inflação de notas é, de facto, real, e tem impacto no acesso ao Ensino Superior? Há uma diferença clara entre escolas privadas e públicas nesta prática, ou é uma cultura em apenas algumas regiões? Fomos analisar os dados disponibilizados pela DGEEC (descrever dados, anos, numero de pares nota interna nota exame) para chegar a uma resposta objetiva a estas questões.

Neste ensaio, apresentamos os primeiros resultados obtidos que mostram uma prática recorrente de inflação de notas em algumas escolas, o que representa uma tremenda injustiça no acesso ao Ensino Superior. Este ensaio apresenta, com detalhe no método, tudo o que concluímos até agora. Partindo daqui, temos dois objetivos: escrever uma versão mais curta e simples e tentar atingir um público geral, por exemplo publicando num jornal, com a ambição de iniciar uma discussão pública sobre este problema que leve à mudança do sistema de acesso ao Ensino Superior; por outro lado, aprofundar a análise aqui apresentada para escrever um artigo a publicar numa revista científica.

Porque não queremos apenas criticar o problema, vamos no fim discutir soluções. Dito isto, é importante enfatizar a diferença de qualidade nas duas partes do texto: na maior parte em que descrevemos o problema, estaremos a mostrar algo factual, que não é uma questão de opinião - existe inflação de notas em muitas escolas em Portugal e este facto é demonstrável. Por outro lado, no final, a discussão de soluções já será uma questão mais de opinião pessoal - aí, queremos apenas lançar uma discussão, e estamos ativamente à procura de mais e melhores ideias do que as nossas. 

Sem mais demoras, começamos com uma introdução técnica que pode ser ignorada por aqueles meramente interessados nos resultados e conclusões. 

\section{Introdução Técnica}

O ponto de partida da nossa análise é a quantidade mais simples e objetiva que podemos imaginar para estudar a inflação de notas, à qual iremos chamar desalinhamento de nota:

\[
\text{Desalinhamento de Nota} = \text{Nota Interna} - \text{Nota Exame Nacional}
\]  

Esta quantidade está apenas bem definida para disciplinas com exame nacional. É uma quantidade que calculámos para todos os pares nota de exame - nota interna na base de dados. Até dizermos o contrário, toda a discussão seguinte está focada apenas nestas disciplinas em que temos uma medida de inflação límpida, porque podemos usar a performance no exame nacional, igual para todos os alunos do país, de qualquer escola, como baseline. 

\textbf{O paradoxo de Simpson entra em cena}

Até agora, tudo simples, no entanto, vamos ter de complicar um bocadinho. O culpado é o muito interessante paradoxo de Simpson, 




\end{document}